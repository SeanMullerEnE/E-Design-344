\chapter{Literature survey}\label{chap:Lit}

This chapter will cover a review on the different methods for current sensing and the limitations, considerations and configurations for operational amplifier design for current sensing. 

\section{Operational amplifiers}\label{sec:opamps}

\subsubsection{Operational amplifiers: limitations and considerations}\label{sec:opamps_limits}
limits that need to be considered for this specific use.
\subsubsection{Operational amplifier configurations}\label{sec:opamps_configs}
inverting
non-inverting
differential

\newpage
\section{Current sensing}\label{sec:cursens}
There are many different techniques to measure current. Both invasive and non invasive methods each with their own advantages and disadvantages that make them suitable for different situations. An invasive current sensor negatively affect the system and decreases performance whereas a non-invasive current sensor doesn't affect the operation of the system at a meaningful level. 

\subsubsection{Hall effect}\label{sec:cursens_hall}
The Hall effect current sensor is a non invasive method of current sensing that uses the magnetic field generated around a current carrying conductor \cite{CircuitDigest}. This magnetic field creates a voltage across the material of the sensor. Hall effect sensors measure this voltage to determine the current flowing in a conductor \cite{Hall}. There are many advantages to using a Hall effect sensor however, besides the amplifier circuit additional circuits are required and is more costly than other measurement methods \cite{CircuitDigest}.

\subsubsection{Rogowski coil}\label{sec:cursens_coil}
The Rogowski coil is a non invasive current sensing method that uses a helical shaped coil that is wrapped around the conductor that you want to measure current in. The coil outputs a voltage depending on the rate of change of current through the conductor, this requires an integrator circuit to create an output voltage that is proportional to the current. The Rogowski coil is very useful for high frequency currents and does not require complex temperature compensation. However this method is only suitable for AC current \cite{CircuitDigest}.

\subsubsection{Shunt Resistor}\label{sec:cursens_shunt}
The shunt resistor is the most common current sensing technique and uses a resistor in series with the current to be measured. This is a invasive current measuring method. The shunt resistor produces a voltage drop proportional to the current, however the resistance and hence the voltage must be kept low in order to reduce the power consumption. This requires a high gain amplifier circuit to increase the small voltages to meaningful levels. The shunt resistor is a very cost effective solution that works on both AC and DC however it creates a decrease in system efficiency and can't handle high currents due to power dissipation across the resistor. Thermal drift also results in error \cite{CircuitDigest}. 

High-side vs low-side current sensing is only applicable for invasive methods like the shunt resistor. Low-side has the advantages of being simple and low cost and low input common mode voltage however it can't detect high current due to a short \cite{EG_CurSens}. High-side current sensing removes the ground disturbance and can detect accidental shorts however it has a higher complexity and cost because the gain circuit must be able to handle high common mode voltages\cite{EG_CurSens}.  